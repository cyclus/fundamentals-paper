
\subsubsection{Discrete Facilities and Materials}

Many fuel cycle phenomena have aggregate system-level effects which can only be
captured by discrete material tracking \cite{huff_next_2010}.  \Cyclus
tracks materials as discrete objects. Some current fuel cycle simulation tools
such as \gls{COSI}
\cite{mccarthy_benchmark_2012,grasso_nea-wpfc/fcts_2009,guerin_benchmark_2009},
\gls{DESAE}
\cite{andrianova_desae_2008}, FAMILY21\cite{mccarthy_benchmark_2012},
\gls{GENIUSv1}, \gls{GENIUSv2}, and \gls{NFCSim} also possess the ability to
model discrete materials.

Similarly, the ability to model disruptions (i.e. facility shutdowns due to
insufficient feed material or insufficient processing and storage capacity) is
most readily captured by software capable of tracking the operations status of
discrete facilities \cite{huff_next_2010}.  Fleet-based models (i.e.
\gls{VISION}) are unable to capture this gracefully.  All of the software
capable of discrete materials have a notion of discrete facilities, however not
all handle disruption in the same manner. \gls{DESAE}, for example, does not
allow shutdown due to insufficient feedstock, though \Cyclus does. In the event
of insufficient fissile material during reprocessing, \gls{DESAE} borrows
material from storage, leaving a negative value \cite{mccarthy_benchmark_2012}.

\subsubsection{Open Access}

The proprietary concerns of research institutions and security constraints of
data within fuel cycle codes often restrict code access. A code's use is
therefore often limited to its institution of origin, necessitating effort
duplication at other institutions and thereby squandering broader human
resources. License agreements and institutional approval are required for most
current codes (i.e. \gls{COSI}6, \gls{DANESS}, \gls{DESAE}, EVOLCODE,
FAMILY21, \gls{NFCSim})\cite{juchau_modeling_2010}, ORION, and VISION.  Even when, as in
the case of the MIT \gls{CAFCA} software, the source code is unrestricted, the
platform on which it relies is oftenrestricted or commercial.  However, \Cyclus
provide fully free and open access to all users
and developers, foreign and domestic.

\subsubsection{Flexibility and Extensibility}

In addition to being inaccessible, the vast majority of fuel cycle simulators
are, effectively, inextensible. Typical fuel cycle simulator evelopment teams are insular, in
part due to accessiblity issues. Also, however, their frameworks contain
heuristics and assumptions that constrain developer additions.

For example, where most other simulators describe a finite set of acceptable
cycle constructions (once through, single-pass, multi-pass), the \Cyclus
simulation logic relies on a market paradigm that can be easily directed to
create novel material and economic paths. The simulation of the dynamic
responses within the system to pricing, availability, and other institutional
preferences in \Cyclus is an advanced capability and does not exist in a
flexible manner for other fuel cycle simulators.

