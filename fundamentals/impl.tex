\section{Methodology and Implementation}
% What were the methods used?
% How was the problem designed?
% Driving concepts
% Equations
% Figures

Only a modular, agent-based approach is capable of adequately solving system 
dynamics problems involving material routing, facility deployment, regional and 
institutional hierarchies.

\subsection{Framework Structure}
% (OO, cpp, xml, backends, inheritances, mixins, generic apis, etc.)

Agent-based modeling is inherently object oriented. 

The core of the simulator creates a set of key classes on which agent plugins 
are based. In addtion, a set of key tools are also provided, to enrich the API 
and provide a robust suite of behaviors for the developer.

<diagram of core, modules, toolkit, etc>

Agent plug-ins utilize the generic core API to interact with one another. 
Mainly they do this by trading resources. 

<diagram of black box facilities>


\subsection{Dynamically loadable libraries}
% (diagram)
\subsection{Agent Interchangability}
% interchangeability due to exchange behavior api
\subsection{Region/Institution/Facility hierarchy}
% (diagram)
\subsection{Discrete Object Tracking}
% Resources, Materials (note isotope tracking, decay behavior)
\subsection{Toolkit}
% library of tools
% contributions to the toolkit
% place for metrics
\subsection{Cycamore}
% base modules
